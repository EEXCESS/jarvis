\section{Evaluation}
 \subsection{1. Evaluation: Interne Evaluation}

 \subsection{2. Evaluation: Thinking Aloud Tests}
Um die Benutzbarkeit der Anwendung zu testen wurden Thinking Aloud Tests durchgeführt. Bei dieser Art der Evaluation werden den Testern Aufgaben gestellt, die sie mit Hilfe der Anwendung lösen müssen. Dabei sollen sie alles was sie denken und alle Entscheidungen die sie treffen laut aussprechen. Sie sollen sagen warum sie auf genau diesen Knopf drücken oder welche Fragen sich ihnen bei der Nutzung stellen. So können Probleme in der Benutzbarkeit aufgedeckt werden.

Vorteile dieser Methode sind, dass nur eine geringe Anzahl von Testern (in diesem Fall 4) gebraucht werden. Sie kann schon in einem frühen Stadium der Entwicklung mit einem Prototypen durchgeführt werden\footnote{Bei der Evaluation von Jarvis war die Implementierung schon abgeschlossen} und hilft dabei, Designfehler früh zu erkennen und zu beheben. Jedoch ist das Feedback der Testuser sehr subjektiv und das Sprechen während des Testens verlangsamt den Benutzer. Da hier die Benutzbarkeit des Systems getestet werden soll - und nicht etwa die Effizienz - sind Thinking Aloud Tests gut geeignet.

 \subsubsection{Testaufbau}
 - einführung
 - jeder tester einzeln
 - verschiedene Wissenslevel
 - filmen
 - Abschließender Fragebogen
  - aufbau (hardware, display, maus, tastatur, display)
  - dataagreement
  - personen beschreiben
  - personen einzeln auswerten (anhang, informell, anonym). hier nur aggregation
  - Fragebogen und Zettel in den Anhang.
 \subsubsection{Auswertung}


 \subsection{3. Evaluation: Expertentest (optional)}
 Nach Einbindung des neuen Recommenders nochmal durch Experten getestet ob es sich verbessert hat.