\section{Konzept}
 Wie auch bei der von EEXCESS entwickelten Extension fiel auch bei Jarvis die Wahl auf eine Chrome Extension. Google Chrome hat mit 40\% den größten Marktanteil unter Web Browsern und durch die Ähnlichkeit der Extension Architektur von Chrome zu anderen Browsern kann die Anwendung im Nachhinein leicht auf diese übertragen werden \cite{schlottererweb}. Per ``JavaScript Injection'' kann in diesen Architekturen das Aussehen und Verhalten der Seite verändert und neue Funktionalität hinzugefügt werden \cite{schlottererweb}.

 Beim Entwurf von Jarvis wurde sich am Design von JITIR-Agents orientiert. Trotzdem sucht Jarvis nicht proaktiv nach Ergebnissen, sondern erst nach einer Interaktion des Users. Zum einen entstehen durch die Nutzung der Europeana API technische Limitierungen. Für jeden API-Key lässt Europeana pro Tag nur 10.000 Anfragen zu \cite{europlimit}. Dieser API-Key wird von allen installierten Jarvis Extensions genutzt. Da in jeder geöffneten Webseite für jeden gefundenen Paragraphen eine eigene Anfrage an die Europeana API gesendet werden würde, wäre dieses Limit ab einer gewissen Anzahl von Nutzern schnell überschritten. Zum anderen kann der User so entscheiden, ob er zum jeweiligen Paragraphen weitere Informationen erhalten möchte oder nicht. Möchte er dass nicht, wird seine Aufmerksamkeit auch nicht durch das erscheinen von Ergebnissen gestört. Der letzte Grund, der zu dieser Entscheidung geführt hat, ist, dass der User sich auf diese Weise seine Suchanfrage selber erstellen kann, bevor diese abgeschickt wird. Entsteht beim Lesen eines Absatzes ein Bedarf auf spezifische Informationen, kann der User die Anfrage direkt über dem Absatz einfügen oder sie interaktiv zusammenbauen. Die Resultate dieser Anfrage haben mit hoher Wahrscheinlichkeit eine höhere Relevanz als die von einer automatisch generierten \cite{rhodes2000just}.

 Der Entwurf der Extension lässt sich in vier Ziele einteilen:
 \begin{enumerate*}
 	\item Die Anzeige der Ergebnisse
  	\item Die Erklärung der Suchanfragen Generierung
 	\item Das Anpassen der Suchanfrage durch den User
 	\item Die Verbesserung der Ergebnisgüte.
\end{enumerate*}
Auf diese Ziele wird im folgenden genauer eingegangen.

 \subsection{Anzeige der Ergebnisse}
 Nachdem eine Anfrage an die Europeana API gesendet wurde und die Ergebnisse beim Client angekommen sind, müssen diese dargestellt werden. Der User soll leicht erkennen können, welche der ihm angezeigten Elemente aus der ursprünglichen Webseite stammen und welche durch die Extension hinzugekommen sind. Aus diesem Grund wurde bei Jarvis ein einheitliches und buntes Design gewählt. So setzt sich die Anwendung deutlich von den meisten Webseiten ab. Auch soll eine enge Bindung zwischen den Ergebnissen und dem dazugehörigen Paragraphen geschaffen werden. Um das zu erreichen, werden die durch die Paragraphen-Erkennung gefundenen Textpassagen hervorgehoben und die Ergebnisse direkt bei den jeweiligen angezeigt.

 Weiterhin ist es wichtig, einen Mittelwert zu finden, was die Menge der angezeigten Informationen betrifft. Zu viele Informationen auf einmal können den User ablenken. Auf der anderen Seite ist es das Ziel dem User eine möglichst reichhaltige alternative Informationsquelle zu liefern \cite{rhodes2000margin}. Rhodes empfiehlt hierfür ein Ramping Interface \cite{rhodes2000just}. Hierbei werden die Informationen in Stufen aufgeteilt. Jede Stufe liefert etwas mehr Informationen als die vorherige. Auf diese Weise kann der User entscheiden ob er mehr Informationen sehen möchte und daraufhin auf die nächste Stufe gehen oder nicht.

 Jarvis zeigt nach einer erfolgreichen Suche zunächst nur die Anzahl der gefundenen Ergebnisse an. Diese Anzeige ist aufgeteilt in Anzahl der textuellen Ergebnisse, der gefundenen Bilder sowie der gefundenen Audio- und Videoquellen. Durch Auswählen einer der drei Kategorien werden in einem sich über dem Paragraphen öffnenden Fenster eine Liste mit Titel und Vorschaubild der Ergebnisse angezeigt. Die Liste ist absteigend nach der Relevanz sortiert, die Europeana in Form eines Relevance Scores mitliefert. Durch ändern des Tabs kann hier zwischen den Kategorien gewechselt werden.
 Durch einen Klick auf den Titel oder das Bild eines Listeneintrages kann der User die entsprechende Quelle aufrufen. Bewegt er bei Text- oder Audio-/Videoquellen den Cursor über das Miniaturbild eines Ergebnisses, werden ihm weitere Informationen angezeigt. Unter anderem die Sprache, die Nutzungsrechte und der Relevance Score der Quelle. Da bei Bildquellen die Metadaten nicht so wichtig sind wie das Bild selber, werden sie in einer Gitter-Liste angezeigt. Die Listen-Elemente bestehen aus einem mittelgroßen Vorschaubild sowie dem Titel des Bildes. Durch klicken gelangt der User zur Quelle des Bildes. Weitere Informationen werden nicht angezeigt, da sie für die Entscheidung die Quelle aufzurufen in den meisten Fällen unwichtig sind.

 \subsection{Erklärung der Suchanfragen Generierung}
 Für den User soll es einfach zu erkennen sein, wie die Ergebnisse zustande gekommen sind. Dafür ist es wichtig, dass er sieht aus welchen Suchbegriffen die Anfrage zusammengesetzt wurde. Nach dem Auslösen der Suchfunktion der Extension werden die Schlagwörter, die aus diesem Paragraphen extrahiert wurden, über diesem angezeigt. Diese Anzeige erlaubt auch die Manipulation der Suchanfrage. Weiterhin werden die gefundenen Schlagwörter im Text hervorgehoben.

 \subsection{Anpassen der Suchanfrage durch den User}
 Da automatisch generierte Suchanfragen nicht immer die gewünschten Ergebnisse liefern, kann der User die Suchanfrage selber erstellen oder die generierte manipulieren. Dazu kann er Wörter aus dem Text hinzufügen oder eigene eingeben. Aus der bestehenden Anfrage kann er unpassende oder mehrdeutige Suchbegriffe entfernen. Durch diese Interaktionsmöglichkeiten soll das Programm so intuitiv wie möglich gestaltet werden um Nutzer mit unterschiedlichen Erfahrungsleveln anzusprechen. Die beschriebenen Funktionalitäten können weiterhin für interaktive Elemente der Query Extension genutzt werden.

 \subsection{Verbesserung der Ergebnisgüte}
 Durch den Einsatz von Methoden des maschinellen Lernens sowie von Query Extension soll die Relevanz der Ergebnisse verbessert werden \cite{ruthven2003re}. Für diesen Anwendungsbereich geeignete Techniken sind Automatic Query Extension\footnote{Automatic Query Extension ist hierbei ein Beispiel für maschinelles Lernen.} (z.B. das Erstellen von Suchprofilen der User \cite{budzik2000user}) und Interactive Query Extension (z.B. Relevance Feedback \cite{harman1988towards}. Bei Query Extension werden, mit oder ohne Benutzerinteraktion, zusätzliche Begriffe zur Suchanfrage hinzugefügt. Da vom User erstellte Suchanfragen meistens nur sehr kurz \cite{ruthven2003re} und dadurch oft nicht eindeutig sind, kann so eine Steigerung der Ergebnisgüte erreicht werden. Durch die hinzugefügten Suchwörter können falsche Ergebnisse, die durch andere Bedeutungen der alten Suchwörter entstanden sind, heraus gefiltert werden.

 Ein Beispiel für Automatic Query Extension ist das Erstellen von Suchprofilen. Das Programm versucht dabei vom Verhalten des Users zu lernen. Dazu werden Informationen über alle Suchvorgänge hinweg gespeichert. Diese Informationen können durch das Bewerten von Seiten durch den Nutzer gesammelt werden oder aus den Seiten extrahiert werden, bei denen der User ein Lesezeichen gesetzt hat \cite{budzik2000user}. Die gewonnenen Informationen werden dann in Schlüsselwörter umgewandelt und an die Suchanfragen angehängt.

 Interactive Query Expansion erlaubt dem User, die initiale Anfrage noch zu erweitern \cite{harman1988towards}. Bei Relevance Feedback zum Beispiel werden dem Benutzer die Ergebnisse seiner ersten Suchanfrage präsentiert. Er kann daraufhin einzelne Quellen als relevant oder irrelevant bewerten. Auf Grund dieser Bewertung wird die ursprüngliche Anfrage um positive und negierte Terme erweitert \cite{budzik2000user}.

 Da das den Rahmen dieser Arbeit jedoch sprengen würde, wird auf die Integration dieser Methoden nur in Kapitel \ref{sec:futureWork} Bezug genommen.