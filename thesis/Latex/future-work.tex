\section{Future Work}
\label{sec:futureWork}
 \subsection{Alternative Algorithmen zur Textanalyse}
 Durchschinittle länge der paragraphen sehr unterschiedlich - >service der damit klar kommt
 \subsection{Implementierung einer automatischen Suchanfragen-Verbesserung durch maschinelles Lernen}
 	\begin{itemize}
 		\item Mehr kontextuelle Informationen miteinbeziehen
 		\item Such-Profil des Nutzers erstellen

 	\end{itemize}
 \subsection{Verbesserung der Ergebnis-Güte}
	\begin{itemize}
		\item durch Query Expansion
		Re-examining the Potential Effectivness of Interactive Query Expansion
Query expansion techniques, e.g. [1,5] aim to improve a user's search by adding new query terms to an existing query. A standart method of performing query expansion is to use relevance information from the user
All or some of these expansion terms can be added to the query either by the user - interactive query expansion - or by the retrieval system - automatic query expansion.
argument for IQE is that interactive query expansion gives more control to the user
We compare the effect of query expansion against no query expansion. how good are different approaches to query expansion? 
All AQE strategies were more likely on average to improve a query than harm it. All techniques improved at least 50\% of the queries where query expansion could make a difference to retrieval effectivness.
query dependet strategy not only improves the highest percentage of queries, is most stable, but also gives the highest average precision over the queries, is most effective
users, especially web searchers, often use very short queries [9]. presenting a list of possible expansion terms is one way to get users to give more information, in the form of query words, to the system.
people can recognize expansion terms that are semantically related to the informatin for which they are seeking and expand the query using these terms

 		\item durch Filtern der Ergebnisse (mehr Präzision da Ausbeute bei JITIR nicht so relevant) Clustering\cite{budzik2000user}
 		\item relevance feedback
 				Towards Interactive Query Expansion:
in an era of online retrieval, it is appropriate to offer guidance to users wishing to improve their initial queries. one form of such guidance could be short lists of suggested terms gathered from feedback, nearest neigbhors, and term variants of original query terms
one method for retrieving more relevant documents is to expand the query terms by using relevance feedback, conflating word stems, and/or adding synonyms form thesaurus. These additional terms allow the query to match documents that contain words which are related to the query, but not actually expressed in it. 
system should be able to help the user modify the query in order to retrieve more relevant documents
The size of windows and the patience of the user require that only a reasonable number of terms be displayed
relevance feedback is an interactive retrieval tool
with the user selecting terms form a small subset of the terms derived from relevance feedback
In an interactive situation, the user will select only those terms deemed useful
it has further been shown that user selection from term vaiants and nearest neighbors of query terms can provide terms for query expansion that improve performance to that comparable with feedback.
	\item word sense disambigation \cite{budzik2000user}
	\end{itemize}
\subsection{Anbindung weiterer Quellen}
	Automatische Auswahl der richtigen QUelle laut selecting task relevant sources
 \subsection{Anpassung der Anwendung auf mobile Nutzung}