NITELIGHT: A Graphical tool:

Query formulation is a key aspect of information retrieval
to support end users with respect to the creation and editing of semantic queries we introduce a graphical tool for semanctic query construction that is based on the SPARQL query language
interactive graphical editing environment
graphical query visualization techniques
query formulation tools should avail themselves of user interaction capabilities that contribute to the efficient design of accurate queries qhile maximally exploiting the power and expressivity provided by the constructs of the target query language.
Nitelight enables users to create SPARQL queries using a set of graphical notations and GUI-Based editing actions.
SPARQL is a semantic query language that exploits the triple-based structure of RDF to perform graph pattern matching and contingent RDF triple assertion. http://www.w3.org/2001/sw/DataAccess/rq23/ "SPARQL Query Language for RDF"
Java-based prototype application called NITELIGHT
interface for graphical query creation
Query Design Canvas is the centerpiece for user interaction and query construction in the NITELIGHT tool

SEWASIE is a graphical query generation environment that co-opts natural language representations and graph-based visualizations of the domain ontology. The user is able to extend and customize an initial query by adding property constraints to selected classes or by replacing classes in the query with another compatible class, such as a subclass or superclass. This process of query refinement is accomplished by selecting terms in the sentential structure of a text-based representation of the query.
https://books.google.de/books?hl=en&lr=&id=rU_onmzozu0C&oi=fnd&pg=PA308&dq=Catarci+An+ontology+based+visual+tool&ots=w6lp3ZSaT2&sig=D_36zelk3eVgOJ_eYV8b1RRcwl4#v=onepage&q&f=false

All VQSs aim to support the user with respect to the deliberate creation of queries


Towards Interactive Query Expansion:
in an era of online retrieval, it is appropriate to offer guidance to users wishing to improve their initial queries. one form of such guidance could be short lists of suggested terms gathered from feedback, nearest neigbhors, and term variants of original query terms
one method for retrieving more relevant documents is to expand the query terms by using relevance feedback, conflating word stems, and/or adding synonyms form thesaurus. These additional terms allow the query to match documents that contain words which are related to the query, but not actually expressed in it. 
system should be able to help the user modify the query in order to retrieve more relevant documents
The size of windows and the patience of the user require that only a reasonable number of terms be displayed
relevance feedback is an interactive retrieval tool
with the user selecting terms form a small subset of the terms derived from relevance feedback
In an interactive situation, the user will select only those terms deemed useful
it has further been shown that user selection from term vaiants and nearest neighbors of query terms can provide terms for query expansion that improve performance to that comparable with feedback.


Re-examining the Potential Effectivness of Interactive Query Expansion
Query expansion techniques, e.g. [1,5] aim to improve a user's search by adding new query terms to an existing query. A standart method of performing query expansion is to use relevance information from the user
All or some of these expansion terms can be added to the query either by the user - interactive query expansion - or by the retrieval system - automatic query expansion.
argument for IQE is that interactive query expansion gives more control to the user
We compare the effect of query expansion against no query expansion. how good are different approaches to query expansion? 
All AQE strategies were more likely on average to improve a query than harm it. All techniques improved at least 50\% of the queries where query expansion could make a difference to retrieval effectivness.
query dependet strategy not only improves the highest percentage of queries, is most stable, but also gives the highest average precision over the queries, is most effective
users, especially web searchers, often use very short queries [9]. presenting a list of possible expansion terms is one way to get users to give more information, in the form of query words, to the system.
people can recognize expansion terms that are semantically related to the informatin for which they are seeking and expand the query using these terms





DBpedia Spotlight:
the standart disambiguation algorithm is based upon cosine similarities and a modification of TD-IDF weights. main phrase spotting algorithm is exact string matching.
 the project has focused initially on the english language.

 phrase spotting: is the task of finding passages in a text that later should be linked to DBpedia entities. stardard phrase spotting algorithm used by DBpedia Spotlight is very light on its NLP demands, only requiring one language-dependent step: tokenization.
 candidates are generated:
 	1. identifying all sequences of capitalized tokens
 	2. identifying all noun phrases, prepositional phrases and multi word phrases
 	3. identifying all named entities

 german is available

 select the best candidates. all candidates below a score threshold are removed.
 score is computed for each spot candidate as a linear combination of features (annotation probability), binary features
 \cite{daiber2013improving}


 system for automatically annotating text documents with DBpedia URIs
 DBpedia spotlight computes scores such as prominence (how many times a resource is mentioned in wikipedia), topical relevance (how close a paragraph is to a DBpedia resource's context) and contextual ambiguity (is there more than one candidate resource with similarly high topical relevance for this surface form in its current context?)

 four stages: !spotting stage recognizes in a sentence the phrases that may indicate a mention of a DBpedia resource. 
 !candidate selection is subsequently employed to map the spotted phrase to resources that are candidate disambiguations for that phrase.
 the !disambiguation stage, in turn, uses the context around the spotted phrase to decide for the best choice amongst the candidates. 
 the annotation can be customized by users to their specific needs through !configuration parameters

 spotting:
 labels gathered from the DBpedia resources which can be seen as community approved surface forms.
 set of labels used used for spotting

 candidate selection:
 map resource names to candidate disambiguations. we use the dbpedia lexicalization dataset for determining candidate disambiguations for each surface form.
 the candidate selection offers a chance to narrow down the space of disambiguation possibilities
 candidate selection phase can also be viewed as a way to pre-rank the candidates for disambiguation before observing a surface form in the context of a paragraph

 disambiguation:
 after selecting candidate resources for each surface form, our system uses the context around the surface forms, e.g. paragraphs, as information to find the most likely disambiguations.
 we modeled dbpedia resource occurences in a Vector Space Model.
 our approach is to rank candidate resources according to the similarity score between their context vectors and the context surrounding the surface form.
 \cite{mendes2011dbpedia}