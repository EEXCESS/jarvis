\section{Einleitung und Motivation}
Suchmaschinen gehören heutzutage zu den meistbesuchten Seiten im Internet. Sie sind der Grundstein für den Zugriff auf Informationen im Internet \cite{budzik2000user}. Google\footnote{\url{https://www.google.com/}} ist mit über 3 Milliarden Suchanfragen pro Tag \cite{medium} nicht nur die beliebteste Suchmaschine, sondern auch die meist besuchte Seite im Internet. Auch die chinesische Suchmaschine Baidu\footnote{\url{http://www.baidu.com/}} ist in den oberen fünf Plätzen vertreten \cite{alexa}. Um so mehr sie an Bedeutung gewinnen, umso größer ist der Bedarf an bessern Suchmaschinen \cite{lawrence2000context}.

Doch der Funktionsumfang klassischer Suchmaschinen ist begrenzt. Sie können den Kontext des Users in ihren Anfragen nicht miteinbeziehen. Weiterhin verlangen sie vom Nutzer, seine gegenwärtige Arbeit einzustellen, die Webseite der Suchmaschine aufzurufen und eine Suchanfrage zu formulieren. So wird die Konzentration auf den eigentlichen Arbeitsschritt gestört. Im Schnitt sind nur 67\% der Suchanfragen an Google erfolgreich \cite{winfuture}. Das heißt in circa einem Drittel der Fälle bleibt der Aufruf der Suchmaschine ohne die erwarteten Erfolge. Zu der verlorenen Zeit addiert sich noch die Zeit, die der Nutzer braucht um sich seine Arbeit wieder ins Kurzzeitgedächtnis zu holen.

Auch sind die Interfaces der Suchmaschinen für Computer leichter zu bedienen als für menschliche Benutzer. Anfragen müssen auf die wichtigsten Schlüsselwörter reduziert werden und die Ergebnisse können maschinell deutlich besser verarbeitet werden als durch den Nutzer \cite{budzik1999watson}. 62\% der von Nutzern erstellen Suchanfragen bestehen aus ein bis zwei Wörtern \cite{jansen2000real}. Kontextuelle Informationen werden dabei ausgelassen und die Anfragen sind oft mehrdeutig \cite{budzik1999watson}.

Um diesen Problemen entgegen zu wirken wurden neue Wege zur Informationsgewinnung entwickelt. Eine Möglichkeit sind so genannte Just-in-time Information Retrieval Agents (JITIR-Agents) \cite{rhodes2000thesis}. Sie beobachten im Hintergrund den Kontext des Users und versuchen aus den so erhaltenen Informationen eine Suchanfrage an eine Datenbank oder ein Recommender-System zu schicken. Die gewonnenen Informationen werden dann möglichst unaufdringlich dem User angezeigt. Er kann sich nun entscheiden diese Informationen genauer zu betrachten oder mit seiner Arbeit fortzufahren. Die kognitive Belastung bleibt hierbei sehr gering. JITIR-Agents reduzieren auf diese Weise enorm den Aufwand Informationen zu finden \cite{rhodes2000just}.

Durch ihre Funktionsweise sind sie jedoch nicht so exakt wie klassische Suchmaschinen, da sie nur ``erraten'' können, was den User gerade interessiert. Wenn ein User einer genaue Vorstellung von der Suchanfrage oder den Ergebnissen hat, haben klassische Suchmaschinen Vorteile gegenüber den JITIR-Agents \cite{rhodes2000just}.

Ein weiteres Problem des Internets ist es, dass die Standards, die die simple und leicht skalierbare Architektur des Netzwerks ermöglichen, den Einsatz von fortgeschritteneren Hypermedia-Technologien verhindern \cite{bouvin1999unifying}. Eine Möglichkeit, das Internet um Funktionen zu erweitern ist Web Augmentation (WA). Diaz \cite{diaz2012understanding} beschreibt WA als den Versuch, statt eine neue Technologie zu entwickeln, neue Funktionalität auf eine gerenderte Webseite zu setzen:
\begin{quote}
In some sense, WA is to the Web what Augmented Reality is to the physical world: layering relevant content/layout/navigation over the existing Web to customize the user experience. \cite{diaz2012understanding}
\end{quote}
Solche WA-Tools interagieren mit einem Web Server, Http-Proxy oder dem Browser der Nutzers, um so Inhalte oder Navigationselemente direkt in die angezeigte Webseite einzufügen. Auf diese Weise erlauben sie es, die limitierten Möglichkeiten des World Wide Webs zu Gunsten des Nutzers anzureichern \cite{anderson1997integrating}.

Ein Beispiel für ein solches WA-Tool sowie für ein JITIR System ist die EEXCESS Chrome Extension\footnote{\url{http://eexcess.eu/results/chrome-extension/}}. Es analysiert die aktuell besuchten Seiten des Nutzers und schlägt ihm auf Grund der erlangten Daten am Rand des Browser-Fensters weiterführende Quellen aus der Europeana-Datenbank\footnote{\url{http://www.europeana.eu/portal/}} vor.

Ziel dieser Bachelorarbeit ist es, auf Basis der EEXCESS Extension eine Chrome Extension zu entwerfen und zu implementieren, welche eine Webseite in einzelne Paragraphen aufteilt und zu jedem dieser Paragraphen kulturelle Inhalte der Europeana-Datenbank vorschlägt. Dabei soll das Design der Extension helfen, die bei der in Kapitel 2 erläuterten Evaluierung der EEXCESS Extension aufgetauchten Probleme zu beheben und Schwachstellen zu verbessern.

Um dieses Ziel umzusetzen, wird unter anderem auf Interactive Query Construction, im Speziellen auf Graphic Query Construction zurückgegriffen. Diese Methoden erlauben es dem User, die anfangs noch groben und kurzen Suchanfragen iterativ zu verfeinern \cite{goldman1999interactive} in dem er neue Schlüsselwörter zur Suche hinzufügt \cite{ruthven2003re}. Im Fall von Graphic Query Construction wird diese Verfeinerung über eine graphische Benutzeroberfläche durchgeführt.

Im anschließenden Teil dieser Arbeit wird auf vergleichbare Technologien eingegangen und ihre Gemeinsamkeiten und Unterschiede zu diesem Projekt. Nachfolgend wird das Konzept und die Implementierung des Projekts beschrieben und das Ergebnis evaluiert. Der sechste Teil der Ausarbeitung widmet sich Möglichkeiten für zukünftige Projekte und Verbesserungen. Zuletzt wird ein Fazit gezogen und die Ergebnisse der Arbeit analysiert.