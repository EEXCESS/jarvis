\section{Konzept}
 \subsection{Warum kein Proactiver JITIR-Agent?}
 	- > API Limitierung und decrease cognitive load 
	- > Benutzer entscheidet ob er weitere Informationen erhalten möchte
	- > Benutzer kann Suchanfrage erst anpassen (Nachteil von Margin Notes (Paper 4)
 \subsection{Anzeige der Ergebnisse}
 \subsection{Erklärung der Such-Anfragen Generierung}
 \subsection{Anpassen der Suchanfrage durch den Nutzer}
 \subsection{Verbesserung der Suchanfrage z.B. durch maschinelles Lernen}

\section{Implementierung}
 \subsection{Verwendung von AngularJS für alle Komponenten des Plugins}
 \subsection{Bau der GUIs}
		- > Darf den Benutzer nicht zu sehr ablenken
		- > Ergebnisse müssen in der Nähe ihrer „Quelle“ angezeigt werden (proximity compatibility pricinple)
		- > Benutzer muss klar zwischen Webseite und Augmentation unterscheiden können
		- > buntes, auffälliges Design
		- > Ramping interface: Mehr Benutzerinteraktionen führen zu mehr angezeigten Informationen (Erklärung der Stages) 
 \subsection{Einbindung der REST-Services}

\section{Evaluierung?}
Aufgaben die Benutzer mit EEXCESS Lösen mussten müssen sie jetzt mit Redesign lösen. Vergleich der Ergebnisse?

\section{Future Work}
 \subsection{Alternative Algorithmen zur Textanalyse}
 \subsection{Implementierung einer automatischen Suchanfragen-Verbesserung durch maschinelles Lernen}
 	\begin{itemize}
 		\item Mehr kontextuelle Informationen miteinbeziehen
 		\item Such-Profil des Nutzers erstellen
 	\end{itemize}
 \subsection{Verbesserung der Ergebniss-Güte}
	\begin{itemize}
		\item durch Query Expansion
 		\item durch Filtern der Ergebnisse (mehr Präzision da Ausbeute bei JITIR nicht so relevant)
	\end{itemize}
 \subsection{Anpassung der Anwendung auf mobile Nutzung}

 \section{Conclusion}
 \begin{itemize}
 		\item Steigerung der Effektivität und Produktivität von wissenschaftlichem Arbeiten
 		\item Starke Effektivitätssteigerung durch Punkte aus Future Work möglich?
 		\item information management assistants embody a vision of a future in which users hardly ever form a query to request information. when an information need arises, a system like watson has already anticipated it and provided relevant information to the user before she is even able to ask for it (budzik, watson)
 		\item survey suggested that users are relatively dissatisfied with the results of their searching experience. THis makes concrete the claim that systems designed to help useres in their information seeking tasks are needed in the world (budzik, watson)
 	\end{itemize}