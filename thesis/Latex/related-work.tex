\section{Related Work}
Rhodes\cite{rhodes2000just} definiert JITIR-Agents als eine Klasse von Programmen, die dem Benutzer weiterführende Informationen basierend auf seinem lokalen Kontext anzeigen. Dabei beschränkt sich der überwachte Kontext meist auf die virtuelle Umgebung des Benutzers, wie E-Mail, Webseiten und geöffnete Dokumente. Als Kerneigenschaften von JITIR-Agents nennt Rhodes Selbstständigkeit, die Fähigkeit Informationen in einer leicht zugänglichen und gleichzeitig unaufdringlichen Weise darzustellen und das Bewusstsein über den Kontext des Benutzers\cite{rhodes2000just}.
Das im Zuge dieser Bachelorarbeit entworfene Programm (im Folgenden als Jarvis bezeichnet) erfordert vom Nutzer die explizite Aufforderung um nach Informationen zu suchen. Gründe für diese Entscheidung werden im dritten Kapitel näher betrachtet. Trotz dieses Widerspruchs zu Rhodes Definition lässt sich Jarvis am besten mit dieser Klasse von Programmen vergleichen.

\subsection{Unterschiede und Gemeinsamkeiten zu JITIR-Agents}

studies in computer response time indicate that small increases in the efoort required to perform a task can have large effects on whether a person will bother acting at all.
miller argues that for may tasks more than two seconds of response delay is unacapptable and will result in fewer uses of a particular tool, even at the cost of decreased accuracy - > robert miller
same short-term memomry limitations miller discusses also apply to performing subtasks that distract from a primary task. for example when performing a search for information about a digressing, a person needs to to use shot-term memeory to keep his or her place in the larger framework of the task. the amount of short-term memory required, and thus the amount of effort required in the task, will depend on a number of factors including the complexity of the digression, the amount of time required to complete the task, and the similarity of the digression to the primary task.
- > applied to the information search domain, these theories suggest that if the cost of finding and using information is more than the expected value of that information, then the search will not be performed.
Jitir agents greatly reduce the costs of searching for information by doing most of the work automatically. the downside is that queries are automatically generated and therefore will not be as exact as would a human-generagted query.

will not bother perform a search because a lack of time. information he has is already "good" enough. person expects a search will not turn up anything useful. rhodes



 \subsubsection{Proaktiv vs. User Interaction}
 \subsubsection{Informationen darstellen in „Nonintrusive Manner“}
 \subsubsection{Awareness of user's local context}
\subsection{Vergleich mit existierenden JITIR Agents}
 	\subsubsection{Remembrance Agent (in EMACS Editor)}
		Zeigt Quellen an auf Basis des geschriebenen Texts, Benutzer kann Suchanfrage dann auch noch manuell anpassen/verfeinern
		- > Vorteil von Suchmaschinen wird mit integriert
 	\subsubsection{Margin? Web Plugin ähnlich wie EEXCESS}
 	\subsubsection{Watson}
 	\subsubsection{EEXCESS?}
 	\subsection{Text Retrieval Algorithmen}
 		Term Frequency/Inverse Document Frequency algorithm
 		Text rank
 \subsection{Unterschiede und Gemeinsamkeiten zu „Automatic help systems“ (z.B. Microsoft Office Assistant - > Domain spezifisch)}
 	Domain spezifisch vs. Domain unabhängig