\section{Related Work}
Rhodes \cite{rhodes2000just} definiert JITIR-Agents als eine Klasse von Programmen, die dem Benutzer weiterführende Informationen basierend auf seinem lokalen Kontext anzeigen. Dabei beschränkt sich der überwachte Kontext meist auf die virtuelle Umgebung des Benutzers, wie E-Mail, Webseiten und geöffnete Dokumente. Als Kerneigenschaften von JITIR-Agents nennt Rhodes Selbstständigkeit, die Fähigkeit Informationen in einer leicht zugänglichen und gleichzeitig unaufdringlichen Weise darzustellen und das Bewusstsein über den Kontext des Benutzers \cite{rhodes2000thesis}.
Das im Zuge dieser Bachelorarbeit entworfene Programm (im Folgenden als Jarvis bezeichnet) erfordert vom Nutzer die explizite Aufforderung um nach Informationen zu suchen. Gründe für diese Entscheidung werden im dritten Kapitel näher betrachtet. Trotz dieses Widerspruchs zu Rhodes Definition lässt sich Jarvis am besten mit dieser Klasse von Programmen vergleichen. Warum diese Klassifizierung zutrifft wird im folgenden Abschnitt beschrieben.

\subsection{Unterschiede und Gemeinsamkeiten zu JITIR-Agents}
Studien haben gezeigt, dass schon eine kleine Steigerung des Aufwands, der betrieben werden muss um eine Aufgabe zu erfüllen, dazu führen kann, dass man die Aufgabe gar nicht erst ausführt \cite{rhodes2000just}. Laut Miller reicht für die meisten Aufgaben eine Antwortverzögerung von mehr als zwei Sekunden aus, um die Nutzungshäufigkeit des dazugehörigen Programms zu vermindern \cite{miller1968response}. Längere Zeitintervalle erschweren es, den Kontext der gerade ausgeführten Aufgabe und die übergeordneten Aufgaben im Kurzeitgedächtnis zu behalten. Nun wird der Fall betrachtet, dass das Lesen einer Webseite unterbrochen wird, um eine Suche mit einer Suchmaschine durchzuführen. Die eigentliche Aufgabe behält der Nutzer im Kurzzeitgedächtnis. Je länger die Suche dauert und je mehr er sich dazu von seiner eigentlichen Arbeit distanzieren muss, desto schwerer wird es wieder zur Hauptaufgabe zurück zu kehren. Wenn der Exkurs zur Suchmaschine schwerer wiegt ist als die Güte der erwarteten Resultate wird die Suche nicht durchgeführt \cite{rhodes2000just}.

Dieses Problem wird versucht mit JITIR-Agents zu beheben. Durch ihre proaktive Arbeitsweise muss der Nutzer seine Tätigkeit nicht mehr unterbrechen, sondern nur noch entscheiden ob er weiter Informationen sehen möchte oder nicht \cite{rhodes2000just}. Beim Entwurf von Jarvis wurde entschieden, dass das Programm nicht völlig eigenständig nach Informationen sucht, sondern nur die Webseite in seine Paragraphen aufteilt. Der Nutzer kann dann entscheiden, ob er zu einem Paragraphen eine Suche durchführen möchte und diese dann per Klick starten. Wie bei einem JITIR-Agent wird die Suchanfrage automatisch aus den gewonnenen Kontextinformationen generiert. Da die Suche innerhalb weniger Millisekunden ausgelöst werden kann und sich der Nutzer dazu nicht von seiner eigentlichen Aufgabe distanzieren muss, bleibt die kognitive Belastung sehr gering. Allerdings entsteht auch so der Nachteil, den Jarvis mit JITIR-Agents gemein hat: Sie nutzen alle gefunden Informationen für ihre Suchanfragen und können nicht zwischen relevanten und unwichtigen Suchwörtern unterscheiden \cite{rhodes2000margin}. Automatisch gebaute Suchanfragen sind deshalb weniger exakt als Menschen-generierte \cite{rhodes2000just}. Um dem entgegen zu wirken hat der Benutzer von Jarvis im Nachhinein noch die Möglichkeit, die Suche anzupassen und erneut abzuschicken.

Die zweite von Rhodes beschriebene Eigenschaft von JITIR-Agents, die Fähigkeit Informationen in einer leicht zugänglichen und gleichzeitig unaufdringlichen Wiese darzustellen, hat auch beim Entwurf von Jarvis eine bedeutende Rolle gespielt. Für die Darstellung der Ergebnisse muss ein Mittelwert gefunden werden. Die zusätzlichen Elemente sollen den Benutzer nicht unnötig ablenken, allerdings will man die Seite um möglichst reichhaltige Informationen erweitern \cite{rhodes2000margin}. Wie diese Problematik im Falle von Jarvis gelöst wurde wird im Implementierungs-Teil dieser Arbeit beschrieben.

Jarvis analysiert die geöffnete Webseite und teilt diese in Paragraphen ein. Wie JITIR-Agents ist er sich also über den lokalen Kontext des Benutzers bewusst und wertet diesen aus. Die dritte Voraussetzung für JITIR-Agents ist somit erfüllt.

Trotz der eingeschränkten Selbstständigkeit lässt sich Jarvis folglich am besten mit denen von Rhodes beschriebenen Programmen vergleichen. Auf diese Erkenntnis aufbauend wird nachfolgend Jarvis mit existierenden Implementierungen von JITIR-Agents verglichen.

\subsection{Vergleich mit existierenden JITIR-Agents}
Es wurden einige JITIR-Agents in der Vergangenheit implementiert, die zwar aus technischer Hinsicht nicht mehr aktuell sind, die sich jedoch durch ihre genaue Untersuchung und Auswertung gut Vergleichen lassen.

 	\subsubsection{Remembrance Agent}
 	Der Remembrance Agent (RA) ist ein in den UNIX Texteditor EMACS-19 integriertes Programm, welches eine Liste von Dokumenten anzeigt, die für den Nutzer interessant seien könnten \cite{rhodes2000thesis}. Die Informationen für die Vorschläge bezieht der RA aus der gerade geöffneten Datei. Wie Jarvis teilt er das Dokument in Bereiche ein, die er danach analysiert. Allerdings sind die Bereiche in diesem Fall keine Paragraphen sondern verschieden große ``Räume'' in denen er sucht. Von der Position des Cursors ausgehend untersucht er die nächsten zehn, die nächsten 50 und die nächsten 1000 Wörter des Dokuments und sucht zu jedem passende Vorschläge \cite{rhodes1996remembrance}. Die Vorschläge werden von einem Information Retrieval Programm im Hintergrund generiert, welches die Suchanfragen an verschiedene, vom Nutzer konfigurierbare Datenbanken schickt. Genannte Datenbanken sind persönliche Email-Verzeichnisse, persönliche Notizen oder Datenbanken mit wissenschaftlichen Arbeiten.

 	Die gefundenen Ergebnisse werden dem Nutzer dann am unteren Fensterrand dargestellt. Mehrere Zeilen präsentieren die Vorschläge mit der höchsten Relevanz. Je nach Art der Quellen werden verschiedene Informationen angezeigt. So werden zum Beispiel bei wissenschaftlichen Arbeiten der Autor, das Datum und der Titel angezeigt und bei Zeitungsartikeln nur Herausgeber, Überschrift und Datum \cite{rhodes2000just}. Dieses Design ermöglicht eine unaufdringliche Darstellung der Vorschläge, die leicht überblickt werden kann aber den Nutzer nicht von seiner Arbeit ablenkt \cite{rhodes1996remembrance}. Per Klick kann sich der Nutzer die Schlagwörter anzeigen lassen, die für das jeweilige Ergebnis verantwortlich sind \cite{rhodes2000thesis}. Mit einer Tastenkombination und der Zeilennummer des Vorschlags kann er sich das dazugehörende, vollständige Dokument anzeigen lassen. Weiterhin hat der Nutzer die Möglichkeit, eine eigene Suchanfrage einzugeben um so explizit nach Inhalten zu suchen \cite{rhodes1996remembrance}.

 	Mit Hilfe einer Nutzer-Evaluation konnte gezeigt werden, dass der RA nicht nur eine Alternative zu klassischen Suchmaschinen darstellt, sondern in vielen Bereichen sogar besser abscheidet. Durch die Nutzung fanden die Tester mehr relevante Dokumente zum geforderten Themengebiet und die anschließende Umfrage schnitt der RA besser ab als die der Kontrollgruppe zur Verfügung gestellte Suchmaschine \cite{rhodes2000just}.

 	\subsubsection{Margin Notes}
 	\subsubsection{Watson}
 	Watson ist ein weiteres Beispiel für JITIR-Agents\footnote{Die Entwickler beschreiben das System als ``Information Management Assistant'', die Bedeutung ist jedoch fast äquivalent.}. Er überwacht die Benutzung von Textverarbeitungs- und Textdarstellungsprogrammen wie Microsoft Word, Microsoft Internet Explorer oder Netscape Navigator \cite{budzik1999watson}. Dazu werden sogenannte ``Application Adapter'' benutzt, die sich mit der jeweiligen Software verbinden um so an den Inhalt der angezeigten oder bearbeiteten Dokumente zu gelangen \cite{budzik2000user}. Die erlangten Informationen werden an Watson weitergegeben, welcher versucht eine passende Quelle aufgrund dieser Daten auszuwählen. Ein weiterer Prozess versucht zu erkennen, ob der Nutzer Bedarf an zusätzlichen Informationen hat. Trifft das zu, wird eine Anfrage an die ausgewählte Quelle/die ausgewählten Quellen geschickt. Die Ergebnisse werden dann gruppiert und in einem separaten Fenster angezeigt \cite{budzik1999watson}.

 	Auch bietet Watson die Möglichkeit, eine Suchanfrage direkt einzugeben. Die Anfrage des Nutzers wird daraufhin mit der automatisch erstellten, kontextabhängigen Anfrage kombiniert um so möglichst relevante Ergebnisse zu liefern \cite{budzik2000user}. Weiterhin werden atomare Bausteine, wie Adressen erkannt. Dem Nutzer wird dann ein Knopf angezeigt, über den er zu einer passenden Darstellung des Bausteins gelangt. Im Falle einer Adresse würde er zum Beispiel zu einer Karte weitergeleitet werden, in der die Adresse angezeigt wird.

 	Evaluationen haben gezeigt, dass die von Watson vorgeschlagenen Dokumente in fünf von zehn Fällen relevant waren. Die Kontrollgruppe, welche die Suche manuell mit Alta Vista\footnote{http://de.wikipedia.org/wiki/AltaVista} durchführen mussten (Watson benutzte als Quelle auch Alta Vista), kamen im Schnitt auf nur drei relevante Ergebnisse. Schlussendlich erzielte Watson bessere Ergebnisse als die Kontrollgruppe in 15 von 19 Fällen \cite{budzik1999watson}.

 	\subsubsection{EEXCESS}

\subsection{Text Retrieval Algorithmen}
 		Term Frequency/Inverse Document Frequency algorithm,
 		Text rank
 \subsection{Unterschiede und Gemeinsamkeiten zu „Automatic help systems“ (z.B. Microsoft Office Assistant - > Domain spezifisch)}
 	Domain spezifisch vs. Domain unabhängig